\documentclass[11pt,a4paper,sans]{moderncv}
\moderncvstyle{banking}                            % style options are 'casual' (default), 'classic', 'oldstyle' and 'banking'
\moderncvcolor{blue}                                % color options 'blue' (default), 'orange', 'green', 'red', 'purple', 'grey' and 'black'
\usepackage[utf8]{inputenc}
\usepackage[scale=0.75]{geometry}
\usepackage{import}

%   ---  Personal data  ---
\name{Julien}{Pilleux}
\title{Informaticien Passionné}
\address{1 Place Jean Cayrol, 33300}{Bordeaux}{France}
\phone[mobile]{06 32 25 72 93}
\email{julien.pilleux@etu.u-bordeaux.fr}
\homepage{http://julien.pilleux.emi.u-bordeaux.fr}

%   ---  Content  ---

\begin{document}
\makecvtitle

%\small{Étudiant en informatique à l'université de Bordeaux en fin de troisième année de licence. Passionné d'informatique}

\section{Formation}

\vspace{5pt}

\begin{itemize}

	\item{\cventry{2014 - 2017}{Licence Informatique}{Université de Bordeaux}{Bordeaux}{}{}}
	\item {\cventry{2017 - 2019 (en cours)}{Master Informatique - spécialité Génie Logiciel}{Université de Bordeaux}{Bordeaux}{}{}}


\end{itemize}

\section{Expérience Professionnelle}

\vspace{6pt}

\begin{itemize}

	\item{\cventry{Juillet 2015 - Aout 2015}{Stage}{Laboratoire Bordelais de Recherche en Informatique}{Bordeaux}{}{\vspace{3pt} Conception d'un logiciel permettant l'anamorphose d'images 2D dans le but d'en faire des hologrammes. Langage utilisé : Java.}}

	\vspace{6pt}

	\item{\cventry{Juillet 2014 - Aout 2014}{Cuisinier}{Quick France}{Bordeaux}{}{}}

\end{itemize}

\section{Compétences}

\vspace{6pt}

\begin{itemize}

	\item \textbf{Langues :} Français (Langue maternelle), Anglais, Italien (niveau A2).

	      \vspace{6pt}

	\item \textbf{Programmation :} C, C++, Java, Python, Bash scripting, LaTeX
	      HTML, CSS, JavaScript.

          \vspace{6pt}

    	\item \textbf{Génie logiciel :} Connaissance des patrons de conception
    		(designe pattern), techniques de refactoring et méthodes de test du logiciel.

	    \vspace{6pt}

    	\item \textbf{Gestion de projet :} Utilisation de la méthode agile Scrum

    	\vspace{6pt}

	\item \textbf{Frameworks :} Bootstrap (css/html/js), Qt (C++), JavaFX (Java)

		\vspace{6pt}

	\item \textbf{Conception formelle :} Introduction en cours de logiciels
	de vérification de modèles comme Rodin ou Altaica

		\vspace{6pt}

	\item \textbf{Logiciels :} Visual Studio Code, Eclipse, Gimp.

	      \vspace{6pt}

	\item \textbf{Systèmes :} Windows XP/Vista/7/8.1/10, Linux Ubuntu/Debian.

	      \vspace{6pt}

	\item \textbf{Autre :} Permis B

\end{itemize}

\end{document}